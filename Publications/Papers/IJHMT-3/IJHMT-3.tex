\documentclass[preprint,12pt]{elsarticle}
\biboptions{sort&compress}
\usepackage{amsmath}
\usepackage{amsfonts}
\usepackage{amssymb}
\usepackage{graphicx}
\usepackage{hyperref}
\usepackage{tabls}
\usepackage{multirow}
\usepackage{cleveref}
\usepackage{verbatim}

\usepackage{pgfplots}
\usetikzlibrary{plotmarks}
\usepackage{tikz}
\usetikzlibrary{shapes,arrows}
\newcommand*{\h}{\hspace{5pt}}% for indentation
\newcommand*{\hh}{\h\h}% double indentation

\usepackage{framed} % Framing content
\usepackage{multicol} % Multiple columns environment
\usepackage{nomencl} % Nomenclature package
\RequirePackage{ifthen}
\renewcommand{\nomgroup}[1]{%
\ifthenelse{\equal{#1}{P}}{\item[\textbf{Superscripts}]}{
\ifthenelse{\equal{#1}{G}}{\item[\textbf{Greek Symbols}]}{
\ifthenelse{\equal{#1}{S}}{\item[\textbf{Subscripts}]}{}}}}

\newcommand{\nomunit}[1]{%
\renewcommand{\nomentryend}{\hspace*{\fill}#1}}
\newcommand{\bv}[1]{\boldsymbol #1}  % change this to change how math vectors are handled

%\usepackage{breqn}

%\special{papersize=8.5in,11in}

\journal{International Journal of Heat and Mass Transfer}


\pdfinfo{%
  /Title(Optimization of an Inverse Convection Solution Strategy)
  /Author   (Yogesh Jaluria)
    /Author   (Joseph R VanderVeer)
  /Creator  (Joseph R VanderVeer)
  /Subject  (Inverse Convection Problems)
}

\begin{document}

\begin{frontmatter}
\title{Optimization of an Inverse Convection Solution Strategy}

\author{Joseph R VanderVeer}
\author{Yogesh Jaluria\corref{cor}}
\ead{jaluria@soemail.rutgers.edu}

\address{Department of Mechanical and Aerospace Engineering: Rutgers University, 98 Brett Rd, Piscataway NJ, 08854}
\cortext[cor]{Corresponding Author}



\begin{abstract}



\end{abstract}
\begin{keyword}
Inverse Problems \sep Computational Heat Transfer \sep Convection
\end{keyword}
\end{frontmatter}

\crefname{equation}{equation}{equations}
\crefname{figure}{figure}{figures}
\crefname{table}{table}{tables}

\newlength\figureheight 
\newlength\figurewidth 
	
	
	
	
\makenomenclature
\setlength{\nomitemsep}{-\parskip} % Baseline skip between items
\renewcommand*\nompreamble{\begin{multicols}{2}}
\renewcommand*\nompostamble{\end{multicols}}
\nomenclature[A]{$T$}{temperature}
\nomenclature[G]{$\bv{\Delta}$}{relative difference between the first sampled point and other sampled points}
\nomenclature[A]{$\bv{r}$}{vector location of sampled points}
\nomenclature[A]{$F$}{minimization function}
\nomenclature[A]{$n$}{number of sample locations}
\nomenclature[G]{$\bv{\delta}$}{vector distance between the actual sampled location and the current test location}
\nomenclature[G]{$\varepsilon$}{error associated with the inverse convection method at a location with given sampled data}
\nomenclature[A]{$d$}{number of simulations}
\nomenclature[A]{$a$}{number of sample locations used in the predictor stage}
\nomenclature[A]{$U$}{free stream velocity}
\nomenclature[A]{$x,y$}{coordinates}
\nomenclature[G]{$\phi$}{normalized temperature $\phi = \frac{T-T_{\infty}}{T_S-T_{\infty}}$}
\nomenclature[A]{$X,Y$}{normalized coordinates}
\nomenclature[S]{$i, j,k$}{index}
\nomenclature[S]{$A,B$}{data set A,B}
\nomenclature[S]{$P$}{predicted}
\nomenclature[S]{$mod$}{modified}
\nomenclature[S]{$\infty$}{free stream}
\nomenclature[P]{$\ast$}{predictor stage, alternative heat flux eqn.}
\nomenclature[S]{$S$}{source}
\nomenclature[A]{$b,m$}{model parameters}
\nomenclature[S]{$0,1,2$}{sample point indexes}
\nomenclature[S]{$O$}{optimized}
\nomenclature[G]{$\rho$}{density}
\nomenclature[A]{$t$}{time}
\nomenclature[A]{$P$}{pressure}
\nomenclature[A]{$E$}{thermal energy}
\nomenclature[G]{$\mu_{t}$}{eddy viscosity}
\nomenclature[A]{$P_{rt}$}{turbulent Prandtl number}
\nomenclature[A]{$k,\epsilon$}{turbulence kinetic energy, dissipation rate}
\nomenclature[A]{$C_1,C_2,C_{1\epsilon},C_{\mu},\sigma_k,\sigma_{\epsilon}$}{$k-\epsilon$ model coefficients}
\nomenclature[A]{$l, I$}{turbulence length scale and intensity}
\nomenclature[G]{$\lambda$}{thermal conductivity}
\nomenclature[G]{$\mu$}{dynamic viscosity}

%\nomenclature[A]{ }{ }

\begin{table*}[!t]
  \begin{framed}
    \printnomenclature
  \end{framed}
\end{table*}




\section{Introduction}
Thermal-fluid systems often create situations where the engineering problem is an inverse heat transfer problem.  These problems often have limited physical access, very limited to no boundary condition knowledge, and/or limited domain knowledge.

For example, an optical fiber drawing furnace wall temperature distribution cannot be directly measured due to shape, inaccessibility, and high temperatures.  \Citet{issa} developed a regularization technique to solve for the distribution using only the furnace centerline temperature distribution.  

\section{Experimental System}

\section{Numerical Simulations}



\section{Inverse Solution Methodology}





\section{Results and discussions}


\section{Conclusions}

\appendix

\bibliography{Bibliography}{}
\bibliographystyle{model1-num-names}
\end{document}

